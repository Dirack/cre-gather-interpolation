\title{Reproducible experiment: Kirchhoff modeling}

\author{Rodolfo A C Neves (Dirack)}
\begin{abstract}
To show how to implement Kirchoff modeling in Madagascar
package. This paper is part of the current directory 
documentation followed by a SConstruct script and a jupyter
notebook explaning details of the script.
\end{abstract}

\section{Introduction}

\indent In this directory we build a seismic data cube from a gaussian reflector in a linear velocity model.
\footnote{This numerical experiment was originaly 
design by Sergey Fomel and Roman Kazinnik in 
the paper Non-hyperbolic common reflection surface
(2013). 
This paper is avaliable as a reproducible paper in
the \href{http://www.ahay.org}{Madagascar website}.}
Velocity increases with depth with a 0.5 
velocity gradient and near surface velocity 
$v_0=1.5Km/s$.

\indent A seismic data cube means that each sample
in the data space can be identifyed by it's CMP, 
half-offset and time coordinates. So, the seismic 
data can be represented by an amplitude function
A(m,h,t), where m is the CMP, h the half-offset
and t is the time.

Kirchhoff modeling \cite[]{km}:

\begin{quote}
Kirchhoff modeling can be defined as the 
mathematical transpose of Kirchhoff migration.
The resulting Kirchhoff modeling algorithm 
has the same low computational cost as 
Kirchhoff migration and, unlike expensive full 
acoustic or elastic wave‐equation methods, 
only models the events that Kirchhoff migration
can image.
\end{quote}

\inputdir{scons}
\plot{gaussianReflectorVelocityModel}{width=\textwidth}{Gaussian reflector model.}

\plot{dataCube}{width=\textwidth}{Seismic data cube.}

\plot{cmpGather}{width=\textwidth}{CMP gather 
extracted from the seismic data cube, 
5Km is the CMP Gather coordinate.}

\bibliographystyle{seg}
\bibliography{mybib}
