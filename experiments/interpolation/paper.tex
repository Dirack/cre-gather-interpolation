\title{Reproducible experiment: PEF interpolation}

\author{Rodolfo A C Neves (Dirack)}
\begin{abstract}
To show how to implement PEF interpolation in Madagascar
package. This paper is part of the current directory 
documentation followed by a SConstruct script and a jupyter
notebook explaning details of the script.
\end{abstract}

\section{Introduction}


\indent In this directory we use a seismic data cube from a gaussian reflector in a linear velocity model. See the numerical experiment "Kirchoff modeling" for details.
\footnote{This numerical experiment was originaly 
design by Sergey Fomel and Roman Kazinnik in 
the paper Non-hyperbolic common reflection surface
(2013). 
This paper is avaliable as a reproducible paper in
the \href{http://www.ahay.org}{Madagascar website}.}
Velocity increases with depth with a 0.5 
velocity gradient and near surface velocity 
$v_0=1.5Km/s$.

\indent Predidive Adaptative Error Filters (PEF) are used to regularize seimic data in this numerical
experiment by interleaving original traces with zeroed traces and interpolating.
The process is done in each constant offset gather extracted from original data with
sfwindow and interleaving zeroed traces in this constant offset gathers to allow interpolation without
changing original data \cite[]{liu}.

\indent For computational reasons we regularize only one constant offset Gather ($h=0$)
and we store it on 'interpolatedDataCube.rsf' file.
Though this process can be done for all constant offset gathers.
You can check the CMP sampling (n2) in 'interpolatedDataCube.rsf'
file using 'sfin' command and you can compare with original CMP sampling in 'dataCube.rsf'.

\begin{verbatim}
~$ sfin interpolatedDataCube.rsf 
interpolatedDataCube.rsf:
    in="/var/tmp/experiments/interpolation/scons/interpolatedDataCube.rsf@"
    esize=4 type=float form=native 
    n1=1001        d1=0.004       o1=0          label1="Time" unit1="s" 
    n2=802         d2=0.0125      o2=0          label2="CMP" unit2="Km" 
    n3=1           d3=0.0125      o3=0          label3="Offset" unit3="Km" 
	802802 elements 3211208 bytes

~$ sfin dataCube.rsf 
dataCube.rsf:
    in="/var/tmp/experiments/interpolation/scons/dataCube.rsf@"
    esize=4 type=float form=native 
    n1=1001        d1=0.004       o1=0          label1="Time" unit1="s" 
    n2=161         d2=0.0125      o2=0          label2="Offset" unit2="Km" 
    n3=401         d3=0.025       o3=0          label3="CMP" unit3="Km" 
	64625561 elements 258502244 bytes
\end{verbatim}

\indent The interpolation process should note change events in the original data. To check that,
see the images of the same constant offset gather $h=0$ in the original and interpolated data in the
Figure~\ref{fig:zeroOffsetGather} and Figure~\ref{fig:interpolatedDataCube}.

\inputdir{scons}
\plot{zeroOffsetGather}{width=\textwidth}{Original data.}
\plot{interpolatedDataCube}{width=\textwidth}{Interpolated data.}

\bibliographystyle{seg}
\bibliography{mybib}
